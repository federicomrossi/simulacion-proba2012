\documentclass{article}
\usepackage[margin=2cm, paperwidth=210mm, paperheight=297mm]{geometry}
\usepackage[spanish]{babel}
\usepackage[utf8]{inputenc}
\usepackage{gensymb}
\usepackage{textcomp}
\usepackage{setspace}
\usepackage{amssymb}
\usepackage{colortbl}


\begin{document}

\title{\huge 61.09 Probabilidad y Estadística \\ 
	  \Huge Trabajo Práctico de Simulación \\
	  \bigskip \Large 13 de mayo de 2012 \\
	  \bigskip \large \textit{Loiza, Samanta (91935)\\Aguilera, Juan Martín (92483)\\Rossi, Federico Martín (92086)}}
\date{}
\maketitle

\section{Introducción}
[ Colocar contenido aquí ]

\section{Actividades previas}

Tal como lo sugiere el título de este apartado, se mostrarán a continuación una serie de actividades previas necesarias para dominar las técnicas básicas de simulación de números aleatorios.

\subsection{Una primera simulación}

A continuación, en el \textit{Código 2.1}, se muestra una rutina que a partir de un generador de números pseudo-aleatorios, permite simular los valores de un dado. Como se puede observar .... [ Colocar contenido aquí ]
\bigskip

\begin{tabular*}{16cm}{l}
	\hline
	\textbf{Código 2.1} \\
	\hline \bigskip
	\texttt{[ Aquí va el código ]} \\
	\hline
\end{tabular*}
\bigskip

\subsection{Puesta a prueba}

Pondremos a prueba ahora lo visto en el apartado anterior estimando las probabilidades de cada uno de los 6 valores posibles utilizando los resultados obtenidos en 1000 ejecuciones de la rutina. [ Colocar contenido aquí ]

\subsection{Predicados}

A veces se necesita estimar la probabilidad de un evento definido a partir de variables aleatorias. [ Colocar contenido aquí ]

\subsection{Enigma final}

Con lo hecho y aprendido hasta ahora podemos plantearnos la misma pregunta que se hizo a si mismo el Caballero de Méré en el siglo XVII: 

\begin{quotation}
\em``¿Cómo puede ser que cuando apuesto a que voy a obtener al menos un doble as en 24 tiradas de dos dados suelo perder, siendo que suelo ganar cuando apuesto a que voy a obtener al menos un as en 4 tiradas?''
\end{quotation}

\noindent [ Colocar contenido aquí ]

\section{Actividad principal}

\end{document}
