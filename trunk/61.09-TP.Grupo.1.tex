\documentclass{article}

%% PAQUETES

% Paquetes generales
\usepackage[margin=2cm, paperwidth=210mm, paperheight=297mm]{geometry}
\usepackage[spanish]{babel}
\usepackage[utf8]{inputenc}
\usepackage{gensymb}

% Paquetes para estilos
\usepackage{textcomp}
\usepackage{setspace}
\usepackage{colortbl}
\usepackage{color}
\usepackage{color}
\usepackage{upquote}
\usepackage{xcolor}
\usepackage{listings}
\usepackage{caption}
\usepackage[T1]{fontenc}
\usepackage[scaled]{beramono}

% Paquetes extras
\usepackage{amssymb}
\usepackage{float}

%% Fin PAQUETES

% Definición de preferencias para la impresión de código fuente.
%% Colores
\definecolor{gray97}{gray}{.97}
\definecolor{gray85}{gray}{.85}
\definecolor{gray75}{gray}{.75}
\definecolor{gray30}{gray}{.30}
\definecolor{bluekeywords}{rgb}{0.13,0.13,1}
\definecolor{greencomments}{rgb}{0,0.5,0}
\definecolor{redstrings}{rgb}{0.9,0,0}

%% Caja de código
\DeclareCaptionFont{white}{\color{white}}
\DeclareCaptionFont{style_labelfont}{\color{black}\textbf}
\DeclareCaptionFont{style_textfont}{\it\color{black}}
\DeclareCaptionFormat{listing}{\colorbox{gray85}{\parbox{16.78cm}{#1#2#3}}}
\captionsetup[lstlisting]{format=listing,labelfont=style_labelfont,textfont=style_textfont}
% Titulo de las cajas de código
\renewcommand{\lstlistingname}{Código}
% Referencia a los códigos
\newcommand{\refcode}[1]{\textit{Código \ref{#1}}}

\lstset{ %
  language = Octave,                
  basicstyle = \footnotesize,       
  numbers = left,                   
  numberstyle = \tiny\ttfamily\color{gray30},  
  stepnumber = 1,                    
  numbersep = 5pt,                 
  tabsize = 2,
  backgroundcolor = \color{gray97}, 
  showspaces = false,               
  showstringspaces = false,         
  showtabs = false,                 
  frame = lines,                   
  rulecolor = \color{gray75},
  tabsize = 2,                      
  captionpos = t,                   
  breaklines = true,                
  breakatwhitespace = true,        
  prebreak = \raisebox{0ex}[0ex][0ex]{\ensuremath{\hookleftarrow}},
  breakatwhitespace = false,
  aboveskip = {1.5\baselineskip},
  columns = fixed,
  upquote = true,
  extendedchars = true,
  title = \null,                    % Default value: title=\lstname;                            
  keywordstyle = \color{bluekeywords}\bfseries,
  commentstyle = \color{greencomments},  
  stringstyle = \color{redstrings},     
  basicstyle = \ttfamily\footnotesize,
  escapeinside = {\%*}{*)},         
}




\begin{document}

% Inserción del título, autores y fecha.
\title{\huge 61.09 Probabilidad y Estadística \\ 
	  \Huge Trabajo Práctico de Simulación \\
	  \bigskip \Large 13 de mayo de 2012 \\
	  \bigskip \large \textit{Loiza, Samanta (91935)\\Aguilera, Juan Martín (92483)\\Rossi, Federico Martín (92086)}}
\date{}
\maketitle


% INTRODUCCIÓN
\section{Introducción}
[ Colocar contenido aquí ]



% ACTIVIDADES PREVIAS
\section{Actividades previas}

Tal como lo sugiere el título de este apartado, se mostrarán a continuación una serie de actividades previas necesarias para dominar las técnicas básicas de simulación de números aleatorios.


% ACTIVIDADES PREVIAS - Una primera simulación
\subsection{Una primera simulación}

A continuación, en el \refcode{code:punto1}, se muestra una rutina que a partir de un generador de números pseudo-aleatorios, permite simular los valores de un dado. Como se puede observar .... [ Colocar contenido aquí ]

% Código
\lstinputlisting[label=code:punto1,caption=randomDado.m]{source/randomDado.m}

% ACTIVIDADES PREVIAS - Puesta a prueba
\subsection{Puesta a prueba}

Pondremos a prueba ahora lo visto en el apartado anterior estimando las probabilidades de cada uno de los 6 valores posibles utilizando los resultados obtenidos en 1000 ejecuciones de la rutina. [ Colocar contenido aquí ]


% ACTIVIDADES PREVIAS - Predicados
\subsection{Predicados}

A veces se necesita estimar la probabilidad de un evento definido a partir de variables aleatorias. [ Colocar contenido aquí ]


% ACTIVIDADES PREVIAS - Enigma final
\subsection{Enigma final}

Con lo hecho y aprendido hasta ahora podemos plantearnos la misma pregunta que se hizo a si mismo el Caballero de Méré en el siglo XVII: 

\begin{quotation}
\em``¿Cómo puede ser que cuando apuesto a que voy a obtener al menos un doble as en 24 tiradas de dos dados suelo perder, siendo que suelo ganar cuando apuesto a que voy a obtener al menos un as en 4 tiradas?''
\end{quotation}

\noindent [ Colocar contenido aquí ]



% ACTIVIDAD PRINCIPAL
\section{Actividad principal}

\end{document}
